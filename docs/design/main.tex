\documentclass{article}
% \usepackage[utf8]{inputenc}
\usepackage[a4paper, margin=1in]{geometry}
% \usepackage{graphicx}
% \usepackage{amsthm,amsmath,amssymb,amsfonts}
% \usepackage{tikz}
% \usepackage{listings}
% \newcommand*\circled[1]{\tikz[baseline=(char.base)]{
%             \node[shape=circle,draw,inner sep=2pt] (char) {#1};}}
% 
% \newcommand{\divides}{\mid}
% \newcommand{\notdivides}{\nmid}
% 
% \usepackage{enumerate} 
% \usepackage{hyperref}
% \usepackage{tabularx}
% \linespread{1.2}
% 
% \usepackage{longtable}
% \usepackage{ltablex}

\usepackage{graphicx}
\usepackage{xltabular}
\usepackage{float}

\usepackage[mode=buildnew,subpreambles=true]{standalone}
\usepackage{tikz}

\usepackage{fancyhdr} % package for headers and footers
\pagestyle{fancy}
\lhead{SAC - }
\rhead{}
% \counterwithin*{equation}{section}

\title{CSC207 Project 1}
\author{SAC}

\date{November 2022}

\begin{document}

\maketitle

\thispagestyle{fancy}

\section{Project Identification}
 The project is intended to create a better version of the Tetris from A2. The outcome of this project contains more function than the tetris from A2 and adopts distinct users' opinions in order to create an extraordinary user experience.
\section{User Story}
% \begin{xltabular}{\textwidth}{|X|X|X|l|l|l|}
\begin{xltabular}{\textwidth}{|p{2cm}|p{0.5cm}|p{1.5cm}|p{3cm}|p{3.5cm}|p{1.75cm}|p{1.25cm}|}
\hline
\textbf{Name}&\textbf{ID} &\textbf{Owner}&\textbf{Description}&\textbf{Implementation Details}&\textbf{Priority}&\textbf{Effort} \\
\hline
Score calculation
& 1.1
& Jianjun Zhao
& As a classic Tetris player, I want a score board which rewards my advanced Tetris skills (like combo, back-to-back, t-spin, etc.) so that I can know how well I played. 
& \textbf{Observer pattern}. The model maintains a list of score observers, who are all notified when a row is cleared. 
& 2 & 1\\
\hline
Hold
& 1.2
& Jianjun Zhao
& As a developer, I want Tetris to give the player flexibility so that they can decide which piece to touch the floor first. 
& Maintain a buffer in model, which allows users to move the current piece to the buffer and move the piece in the buffer to the top of the board. 
& 2 & 2 \\
\hline
Drop buffer (Lock Delay) 
& 1.3
& Junxi Liu
& As a user, I want to perform horizontal movements after a piece touches the floor. 
& Provide some extra time for the user to continue act on the piece after it hits the floor.
& 1 & 3 \\

\hline
Reduce delay 
& 1.4
& Junxi Liu
& As a user, I want Tetris to response with low latency.  
& Tweak JavaFX to increase the framerate.
& 1 & 3 \\

\hline
Color scheme 
& 1.5
& Kaitian Zheng
& As a user who has trouble distinguishing colors due to color blindness, I want to increase the contrast between different blocks so that I can quickly distinguish them. 
& \textbf{Strategy pattern}. Provide an interface ColorScheme which assigns distinct color to each piece. PlacePiece would assign the color generated by current Colorscheme to the piece. 
& 2 & 1 \\

\hline
Ghost Piece
& 2.1
& Shiqi Chen
& As a beginner to Tetris, I want the option to show where a piece will fall to so that the number of misdrops can be reduced. 
& Every time the function `placePiece` is called, place a piece at both the desired position and where the piece will fall (with different color). 
& 3 & 1 \\
\hline
Piece Preview 
& 2.2
& Junxi Liu
& As a classic Tetris player, I want a window showing the upcoming blocks so that I can plan where to drop the blocks in advance. 
& Maintain a queue of `n` upcoming pieces in the model that can be read by view.
& 3 & 1 \\
\hline
T-spin 
& 2.3
& Kaitian Zheng
& As an advanced user, I want to perform special operations like T-Spin so that I can deal with special cases. 
& \textbf{State pattern}. When placePiece is called successfully, it will assign a state to the placed piece, and each state behaves differently when the piece is rotated. 
& 3 & 3 \\
\hline
Game mode 
& 3.1
& Shiqi Chen
& As a player, I want different kinds of game mode (e.g., marathon, 40-lines, time-limited) so that I can have various gaming experiences. 
& \textbf{Template pattern}.
& 3 & 2 \\
\hline
Creative mode (Training mode)
& 3.2
& Kaitian Zheng
& As a beginner to Tetris, I want a training mode where I can follow the guides to practice block setup so that I can become better at Tetris.
& Strategy pattern. Replace the piece generator strategy with a manual one, which provides a window to allow users to choose arbitrary piece.
& 3 & 3 \\
\hline
Input device support [Adaptor] 
& 3.3
& Jianjun Zhao
& As a user, I want Tetris to support various adaptive devices (e.g., joystick) so that I can use devices that satisfy my own need. 
& Adapter pattern. Implement various Adapters that allow communication between this program and different input devices with incompatible interfaces. 
& 3 & 2 \\


\hline
Haptic feedback 
& 3.4
& Kaitian Zheng
& As a developer, I want the user to have a stronger sense of interaction by perceiving the tactile stimuli. 
&
& 3 & 2 \\
\hline
\end{xltabular}
% \end{table}

\newpage

\section{Software Design}

\subsection{Design Pattern 1: Observer}

\paragraph{Overview}
This pattern is used to implement user story 1.1 (Score Calculation).

\paragraph{UML Diagram}\hfill

\begin{figure}[H]
    \includestandalone[width=\linewidth]{score}
\end{figure}

\paragraph{Implementation Details}

\verb`TetrisBoard` maintains a list of \verb`Observers` which observe the behaviour of the \verb`TetrisBoard`. An \verb`Observer` can be added to or removed from the list at any time. Whenever a row is cleared, \verb`TetrisBoard` notifies all its observers to update by calling \verb`notifyAll()`. Once notified, a \verb`ScoreObserver` calculates the score based on its own business logic (e.g., a \verb`ComboScoreObserver` stores the current Combo and rewards players based on the combo they performed). The client can use the \verb`getScore` method to get the score rewarded.

This implementation increases the flexibility of the score calculation scheme. Different kinds of score schemes can be added or removed as needed without modifying the core logic of the client.

\clearpage

\subsection{Design Pattern 2: Template}

\paragraph{Overview}
This pattern is used to implement user story 3.1 (Game mode).

\paragraph{UML Diagram}\hfill

\begin{figure}[H]
    \includestandalone[width=\linewidth]{gamemode}
\end{figure}

\paragraph{Implementation Details}

The class \verb`GameMode` defines how the game runs, by providing four functions \verb`onGameStart()`, \verb`onGameEnd()`, \verb`checkGameEnd()` and \verb`tick()`. The purpose of this class is to separate the driver logic from \verb`TetrisView`. These function could be overwritten to create different game modes, like \verb`TimeLimitedMode`, which ends the game when certain time limit has been arrived, and \verb`GodMode`, which prompts the user to select the piece to be dropped.

\clearpage

\subsection{Design Pattern 3: Strategy}

\paragraph{Overview}
This pattern is used to implement user story 1.5 (Color scheme).

\paragraph{UML Diagram}\hfill

\begin{figure}[H]
    \includestandalone[width=\linewidth]{colorscheme}
\end{figure}

\paragraph{Implementation Details}

\verb`TetrisBoard` maintains a coloring strategy \verb`currentColorScheme`, which contains a concrete
\verb`ColorScheme`. In function \verb`render()`, the concrete color scheme assigns a color to each \verb`Piece` provided. When a new piece is placed on the board (by calling \verb`placePiece()`), a color will be assigned by calling \verb`render()`, and \verb`placePiece()` would assign every block the piece occupies to that color. The users would be able to choose their own color scheme, for example, \verb`DefaultColorScheme` which imitates the classical Tetris colors, and \verb`HighContrastColorScheme`, which is designed for people with vision impairments.

\clearpage

\subsection{Design Pattern 4: State}

\paragraph{Overview}
This pattern is used to implement user story 2.3 (T-Spin).

\paragraph{UML Diagram}\hfill

\begin{figure}[H]
    \includestandalone[width=\linewidth]{spin}
\end{figure}

\paragraph{Implementation Details}

The essence of T-spin implementation is a rotation system that rotates tetrominos differently as per their different states. For example, \verb`WallKickState` represents the state where a basic rotation is obstructed by the wall and the game attempts to rotate the tetrominos in an alternative way. \verb`Piece` stores the reference to one of the concrete \verb`State` and delegates the state-specific rotation function to it. The client can change the rotation method of a tetromino by changing its state using \verb`setState()`. \verb`State` also stores a backreference to the \verb`Piece` object such that \verb`State` can fetch the shape of the \verb`Piece` as well as change the \verb`State` of that \verb`Piece`.

\clearpage

\section{Expected Timeline}

\begin{description}
\item[Week 1] Transition from Assignment 2 starter code, complete refactoring for design patterns and basic functionality.
\item[Week 2] Implement essential gameplay features.
\item[Week 3] Implement advanced gameplay features and accessibility.
\end{description}

\end{document}
