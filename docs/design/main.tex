\documentclass{article}
% \usepackage[utf8]{inputenc}
\usepackage[a4paper, margin=1in]{geometry}
% \usepackage{graphicx}
% \usepackage{amsthm,amsmath,amssymb,amsfonts}
% \usepackage{tikz}
% \usepackage{listings}
% \newcommand*\circled[1]{\tikz[baseline=(char.base)]{
%             \node[shape=circle,draw,inner sep=2pt] (char) {#1};}}
% 
% \newcommand{\divides}{\mid}
% \newcommand{\notdivides}{\nmid}
% 
% \usepackage{enumerate} 
% \usepackage{hyperref}
% \usepackage{tabularx}
% \linespread{1.2}
% 
% \usepackage{longtable}
% \usepackage{ltablex}

\usepackage{graphicx}
\usepackage{xltabular}
\usepackage{float}
\usepackage{hyperref}

\usepackage[mode=buildnew,subpreambles=true]{standalone}
\usepackage{tikz}

\usepackage{fancyhdr} % package for headers and footers
\pagestyle{fancy}
% \counterwithin*{equation}{section}

\title{CSC207 Project 1}
\author{SAC \\ Shiqi Chen, Jianjun Zhao, Junxi Liu, Kaitian Zheng}

\date{November 2022}

\begin{document}

\maketitle

% \thispagestyle{fancy}

\section{Project Identification}
 The project is intended to create a better version of Tetris based on the code from Assignment 2, following the \href{https://tetris.wiki/Tetris_Guideline}{Tetris Design Guideline}. The outcome of this project contains more functions and game modes than Assignment 2 and adopts various users' opinions in order to create an extraordinary user experience.
\section{User Story}
This project uses the MVC pattern. The terms "model" and "view" used below are related to the MVC pattern.
% \begin{xltabular}{\textwidth}{|X|X|X|l|l|l|}
\begin{xltabular}{\textwidth}{|p{2cm}|p{0.5cm}|p{1.5cm}|p{3cm}|p{3.5cm}|p{1.75cm}|p{1.25cm}|}
\hline
\textbf{Name}&\textbf{ID} &\textbf{Owner}&\textbf{Description}&\textbf{Implementation Details}&\textbf{Priority}&\textbf{Effort} \\

\hline Framework & 1.0 & Jianjun Zhao &
As a developer, I want to provide an modular and extensible API for the
Tetris game to made modifications easy. &
Seperate the game logic from Model to a new GameMode class, such that
Model generates event that can be processed by GameMode. &
1 & 3 \\

\hline Score calculation & 1.1 & Shiqi Chen &
As a classic Tetris player, I want a scoring system which recognizes my advanced Tetris skills (like combos, back-to-back, T-spin, etc.) so that I can know how well I played. &
\textbf{Observer pattern}. Maintain a list of score observers which will notify the observers whenever a row is cleared. Then each score observer will calculate the score based on its own score scheme. &
1 & 3 \\

\hline Colour scheme (Accessibility) & 1.4 & Junxi Liu &
As a user who has trouble distinguishing colours due to colour blindness, I want a different colour scheme (like high-contrast) so that I can quickly distinguish GUI elements. &
\textbf{Strategy pattern}. Provide an interface ColorScheme which represents a specific GUI colour scheme. Provide various concrete ColorScheme (like HighContrastColorScheme, GrayscaleColorScheme, etc) to satisfy users with different needs. The game would render the GUI including every Tetromino based on the ColorScheme (strategy) it uses. &
2 & 2 \\

\hline Piece Generator & 1.5 & Kaitian Zheng &
TBD &
TBD &
1 & 1 \\

\hline T-spin & 1.6 & Kaitian Zheng &
As an advanced player, I want to perform special operations like \href{https://tetris.wiki/T-Spin}{T-Spin} (i.e., a mechanism where Tetrominos rotate differently according to their position) so that I can have a more challenging game experience. &
\textbf{State pattern}. Assign every Tetromino a State object. A concrete State contains a different rotation mechanism. A Tetromino can be set to a different State so that its rotation behaviour changes accordingly. &
2 & 3 \\

\hline Game mode & 2.4 & Kaitian Zheng &
As a player, I want different kinds of game modes (e.g., marathon, 40-lines, time-limited) so that I can have various gaming experiences. &
\textbf{Template Method pattern}. Provide a game mode template class which stores default rules. A different game mode inherits this template class and then overrides some of its methods, (like adding a countdown timer to the game-over checker, adding a row-cleared counter, etc). &
2 & 3 \\

\hline Flexible View & 2.5 & Junxi Liu &
TBD &
TBD &
1 & 3 \\


\hline Hold & 3.2 & Junxi Liu &
As a classic Tetris player, I want the game to allow me to store the current Tetromino for later use so that I can decide which piece to touch the floor first. &
Maintain a buffer in the model. The client can store the current Tetromino in the buffer and/or switch the current Tetromino with the one in the buffer at any time. &
2 & 1 \\

\hline Lock Delay & 3.3 & Jianjun Zhao &
As a user, I want the game not to lock the Tetromino once it touches the floor. Instead, I need a lock delay so that I can adjust where to place a Tetromino with more flexibility. &
Set a delay before the program calls the \texttt{lockPiece} method. During the delay, keep the \texttt{keyEvent} listener active so that the user input can still control the game. &
2 & 1 \\

\hline Piece Preview  & 3.4 & Shiqi Chen &
As a classic Tetris player, I want a window showing the upcoming blocks so that I can plan where to place the Tetrominos in advance. &
Maintain a queue of variable length. The client can change the length to change the number of Tetrominos they want to preview. A newly generated Tetromino is enqueued. Dequeue a Tetromino when it is rendered in the view. Render the queue in the view so that users can see the upcoming Tetrominos. &
3 & 3 \\

\hline Ghost Piece & 3.5 & Jianjun Zhao &
As a beginner to Tetris, I want the option to show where a piece will be placed so that the number of mistakenly placed Tetrominos can be reduced. &
Calculate the position where a Tetromino will be placed in advance. Then temporarily land the Tetromino with a fainter colour at the expected position where it will land if allowed to drop. &
1 & 2 \\

\hline Timed Mode & 3.6 & Shiqi Chen &
TBD &
TBD &
3 & 1 \\

\hline Line Count Mode & 3.7 & Junxi Liu &
TBD &
TBD &
3 & 1 \\

\hline Tetris Effect & 3.9 & Jianjun Zhao &
TBD &
TBD &
3 & 1
\hline
\end{xltabular}
% \end{table}

\newpage

\section{Software Design}

\subsection{Design Pattern 1: Observer}

\paragraph{Overview}
This pattern is used to implement user story 1.1 (Score Calculation).

\paragraph{UML Diagram}\hfill

\begin{figure}[H]
    \includestandalone[width=\linewidth]{score}
\end{figure}

\paragraph{Implementation Details}

\verb`TetrisBoard` maintains a list of \verb`ScoreObservers` which observe the behaviour of the \verb`TetrisBoard`. An \verb`ScoreObserver` can be added to or removed from the list at any time. Whenever a row is cleared, \verb`TetrisBoard` notifies all its observers to update by calling \verb`notifyAll()`. Once notified, a \verb`ScoreObserver` calculates the score based on its own business logic (e.g., a \verb`ComboScoreObserver` stores the current Combo and rewards players based on the combo they performed). The client can use the \verb`getScore` method to get the score rewarded. This implementation increases the flexibility of the score calculation scheme. Different kinds of score schemes can be added or removed as needed without modifying the core logic of the client.

\clearpage

\subsection{Design Pattern 2: Template Method}

\paragraph{Overview}
This pattern is used to implement user story 3.1 (Game mode).

\paragraph{UML Diagram}\hfill

\begin{figure}[H]
    \includestandalone[width=\linewidth]{gamemode}
\end{figure}

\paragraph{Implementation Details}

The \verb`GameMode` class decouples the driver logic from \verb`TetrisView`. \verb`GameMode` defines how the game runs, by providing a template method \verb`runGame()` in which four functions \verb`onGameStart()`, \verb`onGameEnd()`, \verb`checkGameEnd()` and \verb`tick()` are called in turn. A concrete \verb`GameMode` class can override these functions to create different game modes. For example, \verb`TimeLimitedMode` adds a countdown timer to \verb`checkGameEnd()`, which ends the game when a certain time limit has arrived. In \verb`TrainingMode`, the \verb`checkGameEnd()` is set to always return true so that players can practice the game without limit. Further, by overriding the \verb`tick()` method, the behaviour of the Tetromino generator is changed so that players can choose which Tetromino they want.

\clearpage

\subsection{Design Pattern 3: Strategy}

\paragraph{Overview}
This pattern is used to implement user story 1.5 (Color scheme).

\paragraph{UML Diagram}\hfill

\begin{figure}[H]
    \includestandalone[width=\linewidth]{colorscheme}
\end{figure}

\paragraph{Implementation Details}

\verb`TetrisBoard` maintains a colouring strategy \verb`currentColorScheme`, which contains a concrete
\verb`ColorScheme`. The method \verb`TetrisBoard.render()`, delegates the render task to \verb`currentColorScheme.render()` which assigns colours to GUI elements. The users would be able to choose their own colour scheme by \verb`setColorScheme()`, for example, \verb`DefaultColorScheme` which follows the classical Tetris colours, and \verb`HighContrastColorScheme`, which is designed for people with vision impairments.

\clearpage

\subsection{Design Pattern 4: State}

\paragraph{Overview}
This pattern is used to implement user story 2.3 (T-Spin).

\paragraph{UML Diagram}\hfill

\begin{figure}[H]
    \includestandalone[width=\linewidth]{spin}
\end{figure}

\paragraph{Implementation Details}

The essence of T-spin implementation is a rotation system that rotates Tetrominos differently according to their different states. For example, \verb`WallKickState` represents the state where a basic rotation is obstructed by the wall and the game attempts to rotate the Tetrominos in an alternative way. \verb`Piece` stores the reference to one of the concrete \verb`State` and delegates the state-specific rotation function to it. The client can change the rotation method of a Tetromino by changing its state using \verb`setState()`. \verb`State` also stores a backreference to the \verb`Piece` object such that \verb`State` can fetch the shape of the \verb`Piece` as well as change the \verb`State` of that \verb`Piece`.

\clearpage

\section{Expected Timeline}

\begin{description}
\item[Sprint 1] Transition from Assignment 2 starter code, complete refactoring for design patterns and basic functionality.
\item[Sprint 2] Implement essential gameplay features.
\item[Sprint 3] Implement advanced gameplay features and accessibility.
\item[Major Milestones] The major milestones will be the implementations of several advanced operations and the optimizations of the feature of Tetris Game.
\end{description}

\begin{figure}[H]
    \includestandalone[width=\linewidth]{timeline}
\end{figure}

\end{document}
