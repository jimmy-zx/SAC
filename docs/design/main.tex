\documentclass{article}
% \usepackage[utf8]{inputenc}
\usepackage[a4paper, margin=1in]{geometry}
\usepackage{listings}
% \usepackage{graphicx}
% \usepackage{amsthm,amsmath,amssymb,amsfonts}
% \usepackage{tikz}
% \usepackage{listings}
% \newcommand*\circled[1]{\tikz[baseline=(char.base)]{
%             \node[shape=circle,draw,inner sep=2pt] (char) {#1};}}
% 
% \newcommand{\divides}{\mid}
% \newcommand{\notdivides}{\nmid}
% 
% \usepackage{enumerate} 
% \usepackage{hyperref}
% \usepackage{tabularx}
% \linespread{1.2}
% 
% \usepackage{longtable}
% \usepackage{ltablex}

\usepackage{graphicx}
\usepackage{xltabular}
\usepackage{float}
\usepackage{hyperref}
\usepackage{ulem}

\usepackage[mode=buildnew,subpreambles=true]{standalone}
\usepackage{tikz}

\usepackage{fancyhdr} % package for headers and footers
\pagestyle{fancy}
% \counterwithin*{equation}{section}

\title{CSC207 Project}
\author{SAC \\ Shiqi Chen, Jianjun Zhao, Junxi Liu, Kaitian Zheng}

\date{December 2022}

\begin{document}

\maketitle

% \thispagestyle{fancy}

\section{Project Identification}

As stated in the phase 1 design document, this project intends to create an advance version of Tetris from Assignment 2, following the \href{https://tetris.wiki/Tetris_Guideline}{Tetris Design Guideline}. While the main goal does not differ significantly, our focus gradually shifted to creating a highly customizable and extensible Tetris. Users will be provided with various options to customize their own game.

\section{User Story}
This project uses the MVC pattern. The terms "model", "view" and "controller" used below are related to the MVC pattern.

% \begin{xltabular}{\textwidth}{|X|X|X|l|l|l|}
\begin{xltabular}{\textwidth}{|p{2cm}|p{0.5cm}|p{1.5cm}|p{3cm}|p{3.5cm}|p{1.75cm}|p{1.25cm}|}
\hline
\textbf{Name}&\textbf{ID} &\textbf{Owner}&\textbf{Description}&\textbf{Implementation Details}&\textbf{Priority}&\textbf{Effort} \\

\hline Framework & 1.0 & Jianjun Zhao &
As a developer, I want to provide a modular and extensible API so that it would be easy to maintain the code and add new features without changing the core logic. &
Subdivide the Model into several subsystems including  \verb|GameMode|, \verb|RotationSystem|, \verb|Generator|, etc.  The Model only communicates with abstract interfaces such that all add-on features are easily replaceable. &
1 & 3 \\

\hline Score calculation & 1.1 & Shiqi Chen &
As a classic Tetris player, I want a scoring system which recognizes my advanced Tetris skills (like combos, back-to-back, T-spin, etc.) so that I can know how well I played. &
\textbf{Observer pattern}. \verb|ScoreSystemLayer| would maintain a list of \verb|ScoreObserver|'s which are notified whenever the Model ticks (advances a step). Then each score observer will calculate the score based on its own score scheme.&
1 & 3 \\

\hline \sout{Hold} & \sout{1.2} & & & Delayed to 3.2 & &  \\

\hline \sout{Piece Preview} & \sout{1.3} & & & Delayed to 3.4 & &  \\

\hline Colour scheme (Accessibility) & 1.4 & Junxi Liu &
As a user who has trouble distinguishing colours due to visual impairment such as colour blindness, I want a different colour scheme (like high-contrast) so that I can quickly distinguish GUI elements. &
\textbf{Strategy pattern}. Provide an interface \verb|ColorScheme| which represents a specific GUI colour scheme. Provide various concrete \verb|ColorScheme| (like \verb|HighContrastColor|, \verb|ClassicColor|, etc) to satisfy users with different needs. The View would render the GUI including every \verb|Piece| based on the \verb|ColorScheme| (strategy) it uses. &
2 & 2 \\

\hline Piece Generator & 1.5 & Kaitian Zheng &
As an advanced player and developer, I want to facilitate different kinds of piece generators. In this way, in some special game modes, we are able to make a perfect row clear according to the behind-scene algorithm. &
\textbf{Strategy Pattern} Maintain an extensible \verb|Generator| interface which enables the later modifications such as adding a new piece generator system. &
1 & 1 \\

\hline T-spin & 1.6 & Kaitian Zheng &
As an advanced player, I want to perform special operations like \href{https://tetris.wiki/T-Spin}{T-Spin} (i.e., a mechanism where tetris pieces rotate differently according to their position) so that I can have a more challenging game experience. &
\textbf{State pattern}. Assign every \verb|Piece| a \verb|RotationState| object. A concrete State contains a specific rotation mechanism. A \verb|Piece| can be set to a different State so that its rotation behaviour changes accordingly. &
2 & 3 \\

\hline Game mode & 2.4 & Kaitian Zheng &
As a player, I want different kinds of game modes (e.g., marathon, 40-lines, time-limited) so that I can have various gaming experiences. &
\textbf{Mediator pattern} Provide two interfaces, \verb|GameCore| and \verb|GameLayer|. The \verb|GameCore| interface is a template for unique features in a Tetris game, such as piece generator and rotation system. The \verb|GameLayer| interface is a template for some stackable additional features, such as the score calculation system and timed mode. A mediator class \verb|GameMode| is provided to aggregate the \verb|GameCore| and a stack of \verb|GameLayer|. &
2 & 3 \\

\hline Flexible View & 2.5 & Junxi Liu &
As a UI developer, I'd like to implement a method to design and control different boxes conveniently in the user interface. &
\textbf{Template pattern} Provide an abstract class \verb|FloatController|. The client can add arbitray floating element by inheriting \verb|FloatController|, and the \verb|FloatController| will take care for adding the floating element into the main view. &
1 & 3 \\


\hline Hold & 3.2 & Junxi Liu &
As a classic Tetris player, I want the game to allow me to store the current \verb|Piece| for the later use so that I can decide which tetris piece to touch the floor first. &
Maintain a buffer in the model. The client can store the current \verb|Piece| in the buffer and/or switch the current \verb|Piece| with the one in the buffer at any time. &
2 & 1 \\

\hline Lock Delay & 3.3 & Jianjun Zhao &
As a user, I want the game not to lock the tetris piece once it touches the floor. Instead, I need a lock delay so that I can adjust where to place the tetris piece with more flexibility. &
When a tetris piece reaches bottom and is not a hard drop, start a \verb|Lock| timer that expires in a specific interval. When the next tick is reached, the user does not perform any movement, and the \verb|Lock| timer expired, then the current tetris piece will be fixed and a new piece is generated. 
2 & 1 \\

\hline Piece Preview  & 3.4 & Shiqi Chen &
As a classic Tetris player, I want a window showing the upcoming blocks so that I can plan where to place the tetris pieces in advance. &
Maintain a queue of variable length. The client can modify that length to change the number of tetris pieces they want to preview. A newly generated \verb|Piece| is enqueued. Dequeue a \verb|Piece| when it is rendered in the view. Render the queue in the view so that users can see the upcoming tetris pieces. &
3 & 3 \\

\hline Ghost \verb|Piece| & 3.5 & Jianjun Zhao &
As a beginner to Tetris, I want the user interface to show where a piece will be landed so that the number of mistakenly placed tetris pieces can be reduced. &
Calculate the position where a tetris piece will be placed in advance. Then temporarily land the tetris piece with a fainter colour at the expected position where it will land if allowed to drop. &
1 & 2 \\

\hline Timed Mode & 3.6 & Shiqi Chen &
As an advanced player, I want to increase the difficulty level of playing Tetris so that the point we get from placing the tetris piece is also related to the time we spent in each interval. &
Add a new \verb|GameLayer| to calculate the time elapsed and check if a specified target of time has elapsed. The elapsed time will be reflected on the main view, and this layer will end the game once a target time is reached. &
3 & 1 \\

\hline Line Count Mode & 3.7 & Junxi Liu &
As a user, I want to benchmark the time required for me to clear a certain number of lines such that I can compete with other players.&
Add a \verb|GameLayer| to record the number of lines cleared and check if a specified target is reached. The number of lines cleared will be displayed on the main view, and this layer will end the game once the target number is reached. &
3 & 1 \\

\hline Tetris Effect & 3.9 & Jianjun Zhao &
As a player, I want Tetris pieces to have some special visual effects (like shadowing or blooming) such that the game would look fancy and more attractive. & Use \verb|javafx.GraphicsContext| to draw Tetris pieces. \verb|javafx.GraphicsContext| can take \verb|Effect| objects like \verb|Bloom| or \verb|DropShadow| as parameters.
&
3 & 1 \\
\hline
\end{xltabular}
% \end{table}

\newpage

\section{Software Design}

\subsection{Design Pattern 1: Observer}

\paragraph{Overview}
This pattern is used to implement user story 1.1 (Score Calculation).

\paragraph{UML Diagram}\hfill

\begin{figure}[H]
    \includestandalone[width=\linewidth]{score}
\end{figure}

\paragraph{Implementation Details}

\verb`ScoreSystemLayer` maintains a list of \verb`ScoreObservers` which observe the behaviour of the \verb`Model`. An \verb`ScoreObserver` can be added to or removed from the list at any time. Whenever the \verb`Model` updates, \verb`ScoreSystemLayer` will notify all its observers and passes them a \verb|DataPackage| (which contains all necessary information) by calling \verb`ScoreObserver.notifyAllObservers()` and. Once notified, a \verb`ScoreObserver` calculates the score based on its own business logic (e.g., a \verb`ComboScoreObserver` stores the current Combo and rewards players based on the combo they performed). The client can use the \verb`ScoreSystemLayer.getScore()` method to get the aggregate score rewarded. This implementation increases the flexibility of the score calculation scheme. Different kinds of score schemes can be added or removed as needed without modifying the core logic of the client.

\clearpage

\subsection{Design Pattern 2: Template Method}

\paragraph{Overview}
This pattern is used to implement user story 2.5 (Flexible View).

\paragraph{UML Diagram}\hfill

\begin{figure}[H]
    \includestandalone[width=\linewidth]{FloatController}
\end{figure}

\paragraph{Implementation Details}

The \verb`FloatController` is an abstract template class for controllers for floating elements. The user is able to create concrete controllers for floating elements, such as \verb`HoldController`, \verb`LineCountController`, and etc independently, without bothering to consider the element's location in the stage.

\clearpage

\subsection{Design Pattern 3: Strategy}

\paragraph{Overview}
This pattern is used to implement user story 1.5 (Color scheme).

\paragraph{UML Diagram}\hfill

\begin{figure}[H]
    \includestandalone[width=\linewidth]{colorscheme}
\end{figure}

\paragraph{Implementation Details}

\verb`MainController` maintains a colouring strategy \verb`colorScheme`. \verb`MainController` delegates its color rendering task to \verb`colorScheme.render()` which assigns a colour to each GUI element. The client would be able to choose their own colour scheme by calling \verb|setColorScheme()|. An example of concrete \verb|ColorScheme| would be \verb`HighContrastColorScheme` designed for people with vision impairments, which makes GUI elements easily distinguishable.

\clearpage

\subsection{Design Pattern 4: State}

\paragraph{Overview}
This pattern is used to implement user story 2.3 (T-Spin).

\paragraph{UML Diagram}\hfill

\begin{figure}[H]
    \includestandalone[width=\linewidth]{spin}
\end{figure}

\paragraph{Implementation Details}

The essence of the T-spin implementation is a rotation system that rotates tetris pieces differently according to their different states. For example, \verb`SuperRotationState` represents the state where a basic rotation is obstructed by the wall and the game attempts to rotate the Pieces in an alternative way. \verb`Model` stores the reference to one of the concrete \verb`RotationState` and delegates the state-specific rotation function to it. The client can change the rotation method of a \verb|Model| by changing its state using \verb`setRotationState()`. \verb`RotationState` also stores a backreference to the \verb`Model` object such that \verb`RotationState` can change the \verb`currentState` of \verb`Model`.

\clearpage

\subsection{Design Pattern 5: Mediator}

\paragraph{Overview}
This pattern is used to implement user story 2.4 (Game mode)

\paragraph{UML Diagam}\hfill

\begin{figure}[H]
\centering
\includestandalone[width=0.7\linewidth]{gamemode}
\end{figure}

\paragraph{Implementation Details}

The game logic was split into two parts: a \verb|GameCore| and a stack of \verb|GameLayer|s. A \verb|GameCore| contains a logic that is unique for each game, such as the piece generators, and the rotation system. A \verb|GameLayer| contains additional features of the game, such as score calculation, line counts, and etc. A \verb|GameMode| implements both \verb|GameCore| and \verb|GameLayer| and contains a \verb|GameCore core| and a \verb|Stack<GameLayer> layers|. It implements \verb|GameCore| by passing the call directly to \verb|core|, and implements \verb|GameLayer| by aggregating the result of each layer in \verb|layers|. In short, \verb|GameMode| serves as a mediator that allows the \verb|Model| to send events to every \verb|GameCore| and \verb|GameLayer| without knowing their implementation details.

\clearpage

\section{Expected Timeline}

\begin{description}
\item[Sprint 1] Transition from Assignment 2 starter code, complete refactoring for design patterns and basic functionality.
\item[Sprint 2] Release a version that functions similarly to a classic Tetris game, with various options for users to customize.
\item[Sprint 3] Merge all branches and resolve any conflict or bug. Finish code documentations, sanity tests, and the report.
\item[Major Milestones] The major milestones will be the implementations of several advanced operations such as T-spin and the optimizations of the feature of Tetris Game.
\end{description}

\begin{figure}[H]
    \includestandalone[width=\linewidth]{timeline}
\end{figure}

\end{document}
